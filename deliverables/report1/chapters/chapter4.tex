%!TEX root = ../mieic.tex

\chapter{Plano de Trabalho}\label{chap:chap4}

\section*{}

Este capítulo pode ser dedicado à apresentação de detalhes de nível
mais baixo relacionados com o enquadramento e implementação das
soluções preconizadas no capítulo anterior.
Note-se no entanto que detalhes desnecessários à compreensão do
trabalho devem ser remetidos para anexos.

Dependendo do volume, a avaliação do trabalho pode ser incluída neste
capítulo ou pode constituir um capítulo separado.

\section{Tarefas a Realizar} % (fold)
\label{sec:tarefas}

  \subsection{Tarefa 1} % (fold)
  \label{sub:tarefa_1}
  
  % subsection tarefa_1 (end)

  \subsection{Tarefa 2} % (fold)
  \label{sub:tarefa_2}
  
  % subsection tarefa_2 (end)

% section tarefas (end)

\section{Avaliação e Validação dos Resultados} % (fold)
\label{sec:avaliacao}

% section avaliacao (end)

\section{Calendarização} % (fold)
\label{sec:calendarizacao}

% section calendarizacao (end)

\section{Resumo ou Conclusões}

\lipsum[1]
