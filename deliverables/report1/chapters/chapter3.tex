%!TEX root = ../mieic.tex

\chapter{Projeto}
\label{chap:chap3}

\section*{}

O principal objectivo desta dissertação, como foi referido no capítulo \ref{chap:intro}, é desenvolver um (ou mais) módulo(s) de software que contribuam para uma melhoria na descoberta e recomendação de música num ambiente integrado entre o RAMA e o Spotify, por forma a tirar partido da representação gráfica do grafo de artistas de música do RAMA e da qualidade do serviço de \emph{Streaming} de música do Spotify.

Para tal, a proposta inicial desta dissertação consiste em desenvolver, no mínimo, um módulo que implemente uma das seguintes funcionalidades:

\begin{enumerate}
  \item \label{item:obj1} Integrar o serviço de \emph{streaming} de música do Spotify \textbf{no RAMA}
  \item \label{item:obj2} Integrar informação de um utilizador Spotify \textbf{no RAMA}
  \item \label{item:obj3} Melhorar design e funcionalidades \textbf{do RAMA}
  \item \label{item:obj4} Integrar a visualização de grafos de artistas de música \textbf{numa Aplicação Spotify}
  \item \label{item:obj5} Integrar o módulo de criação de \emph{playlists} do RAMA \textbf{numa Aplicação Spotify}
  \item \label{item:obj6} Integrar alguns dos módulos acima referidos \textbf{numa aplicação móvel}
\end{enumerate}

As três primeiras funcionalidades (\ref{item:obj1}, \ref{item:obj2} e \ref{item:obj3}) focam-se em melhorar o serviço do RAMA, usando API's do Spotify, ou seja, integrar o Spotify dentro do RAMA.
Por outro lado, as funcionalidades \ref{item:obj4} e \ref{item:obj5} têm como objetivo integrar o RAMA dentro do Spotify, através de uma Aplicação Spotify, que funciona como plugin do programa principal do Spotify.
A última funcionalidade (\ref{item:obj6}) teria de implementar algumas das anteriores num Sistema Operativo Móvel (Android, iOS ou Windows Phone).

Este capítulo procura analisar todas as condicionantes que afetam a escolha  dos módulos a desenvolver, e em que ambientes estes se encaixam melhor (Aplicação Spotify, aplicação móvel ou RAMA).

Inicialmente será explorado o ambiente de desenvolvimento do Spotify, seguido das tecnologias que suportam os referidos módulos, da arquitetura subjacente a cada módulo, e seguida de experimentações feitas que ajudaram na tomada de decisão final.

No final, deve ficar claro quais serão os módulos de software a desenvolver, tendo em conta que o seu objetivo é contribuir para uma melhoria na descoberta e recomendação de música num ambiente relacionado com o RAMA.


\section{Spotify} % (fold)
\label{sec:spotify}


% section spotify (end)

\section{Tecnologias} % (fold)
\label{sec:tecnologias}

% section tecnologias (end)

\section{Arquitetura} % (fold)
\label{sec:arquitetura}


% section arquitetura (end)

\section{Experimentação Feita} % (fold)
\label{sec:experimentacao}

% section experimentacao (end)

\section{Resumo e Conclusões}


