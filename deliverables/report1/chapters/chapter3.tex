%!TEX root = ../mieic.tex

\chapter{Projeto}
\label{chap:chap3}

\section*{}

O principal objectivo desta dissertação, como foi referido no capítulo \ref{chap:intro}, é integrar RAMA e Spotify, desenvolvendo um (ou mais) módulo(s) dos seguintes:

\begin{itemize}
  \item Integrar o serviço de \emph{streaming} de música do Spotify no RAMA
  \item Integrar a visualização de grafos de artistas de música num Aplicação Spotify
  \item Integrar o módulo de criação de \emph{playlists} do RAMA numa Aplicação Spotify
  \item Integrar informação de um utilizador Spotify no RAMA, por forma a melhorar as recomendações de artistas de música
  \item Integrar alguns dos módulos acima referidos numa aplicação móvel
  \item Melhorias ao design e funcionalidades do RAMA
\end{itemize}

No final, deverão ter sido desenvolvidos módulos de software que contribuam para uma melhoria na descoberta e recomendação de música num ambiente relacionado com o RAMA.



\emph{ Este capítulo deve começar por fazer uma apresentação detalhada do
problema a resolver podendo mesmo, caso se justifique, constituir-se um capítulo com essa finalidade.
Deve depois dedicar-se à apresentação da solução sem detalhes de
implementação. 
Dependendo do trabalho, pode ser uma descrição mais teórica, mais arquitetural, etc. }


\section{Spotify} % (fold)
\label{sec:spotify}


% section spotify (end)

\section{Tecnologias} % (fold)
\label{sec:tecnologias}

% section tecnologias (end)

\section{Arquitetura} % (fold)
\label{sec:arquitetura}


% section arquitetura (end)

\section{Experimentação Feita} % (fold)
\label{sec:experimentacao}

% section experimentacao (end)

\section{Resumo e Conclusões}

Resumir e apresentar as conclusões que se podem tirar no fim deste
capítulo.
