%%========================================
%% Spotify-ed: Music Recommendation and 
%%              in Spotify
%%
%% Author: José Bateira
%% Supervisor: Fabien Gouyon
%%========================================

\documentclass[11pt,a4paper,twoside,openright]{report}

%% For iso-8859-1 (latin1), comment next line and uncomment the second line
\usepackage[utf8]{inputenc}
\usepackage{subcaption}


\usepackage{listings}
\usepackage{color}


\definecolor{mygreen}{rgb}{0,0.6,0}
\definecolor{mygray}{rgb}{0.5,0.5,0.5}
\definecolor{mymauve}{rgb}{0.58,0,0}

\lstset{ %
  backgroundcolor=\color{white},   % choose the background color; you must add \usepackage{color} or \usepackage{xcolor}
  basicstyle=\footnotesize,        % the size of the fonts that are used for the code
  breakatwhitespace=false,         % sets if automatic breaks should only happen at whitespace
  breaklines=true,                 % sets automatic line breaking
  captionpos=b,                    % sets the caption-position to bottom
  commentstyle=\color{mygreen},    % comment style
  deletekeywords={...},            % if you want to delete keywords from the given language
  escapeinside={\%*}{*)},          % if you want to add LaTeX within your code
  extendedchars=true,              % lets you use non-ASCII characters; for 8-bits encodings only, does not work with UTF-8
  frame=single,                    % adds a frame around the code
  keepspaces=true,                 % keeps spaces in text, useful for keeping indentation of code (possibly needs columns=flexible)
  keywordstyle=\color{blue},       % keyword style
  language=Octave,                 % the language of the code
  morekeywords={*,...},            % if you want to add more keywords to the set
  numbers=left,                    % where to put the line-numbers; possible values are (none, left, right)
  numbersep=5pt,                   % how far the line-numbers are from the code
  numberstyle=\tiny\color{mygray}, % the style that is used for the line-numbers
  rulecolor=\color{black},         % if not set, the frame-color may be changed on line-breaks within not-black text (e.g. comments (green here))
  showspaces=false,                % show spaces everywhere adding particular underscores; it overrides 'showstringspaces'
  showstringspaces=false,          % underline spaces within strings only
  showtabs=false,                  % show tabs within strings adding particular underscores
  stepnumber=2,                    % the step between two line-numbers. If it's 1, each line will be numbered
  stringstyle=\color{mymauve},     % string literal style
  tabsize=2,                       % sets default tabsize to 2 spaces
  title=\lstname                   % show the filename of files included with \lstinputlisting; also try caption instead of title
}


%\usepackage[latin1]{inputenc}

%% Portuguese version

%% MIEIC options
\usepackage[portugues,mieic]{styles/feupteses}
%\usepackage[portugues,mieic,juri]{feupteses}
%\usepackage[portugues,mieic,final]{feupteses}
%\usepackage[portugues,mieic,final,onpaper]{feupteses}

%% Uncomment the next lines if side by side graphics used
%\usepackage[lofdepth,lotdepth]{subfig}
%\usepackage{graphicx}
%\usepackage{float}

\usepackage{lipsum}
\usepackage{color}
\definecolor{cloudwhite}{cmyk}{0,0,0,0.025}

%% Include source-code listings package
\usepackage{listings}

\lstset{ %
 language=C,                        % choose the language of the code
 basicstyle=\footnotesize\ttfamily,
 keywordstyle=\bfseries,
 numbers=left,                      % where to put the line-numbers
 numberstyle=\scriptsize\texttt,    % the size of the  s that are used for the line-numbers
 stepnumber=1,                      % the step between two line-numbers. If it's 1 each line will be numbered
 numbersep=8pt,                     % how far the line-numbers are from the code
 frame=tb,
 float=htb,
 aboveskip=8mm,
 belowskip=4mm,
 backgroundcolor=\color{cloudwhite},
 showspaces=false,                  % show spaces adding particular underscores
 showstringspaces=false,            % underline spaces within strings
 showtabs=false,                    % show tabs within strings adding particular underscores
 tabsize=2,	                    % sets default tabsize to 2 spaces
 captionpos=b,                      % sets the caption-position to bottom
 breaklines=true,                   % sets automatic line breaking
 breakatwhitespace=false,           % sets if automatic breaks should only happen at whitespace
 escapeinside={\%*}{*)},            % if you want to add a comment within your code
 morekeywords={*}
}

%% Uncomment next line to set the depth of sectional units listed in the toc
%\setcounter{tocdepth}{3}

%\makeindex

\graphicspath{{figures/}}

%% macros
%!TEX root = ../mieic.tex

%some macro definitions

% format
\newcommand{\class}[1]{{\normalfont\slshape #1\/}}

% entities
\newcommand{\Feup}{Faculdade de Engenharia da Universidade do Porto}

\newcommand{\svg}{\class{SVG}}
\newcommand{\scada}{\class{SCADA}}
\newcommand{\scadadms}{\class{SCADA/DMS}}


% listings


\usepackage{listings}

\usepackage{xcolor}

\definecolor{lightgray}{rgb}{0.95, 0.95, 0.95}
\definecolor{darkgray}{rgb}{0.4, 0.4, 0.4}
\definecolor{editorGray}{rgb}{0.95, 0.95, 0.95}
\definecolor{editorOcher}{rgb}{1, 0.5, 0} % #FF7F00 -> rgb(239, 169, 0)
\definecolor{editorGreen}{rgb}{0, 0.5, 0} % #007C00 -> rgb(0, 124, 0)
\definecolor{orange}{rgb}{1,0.45,0.13}    
\definecolor{olive}{rgb}{0.17,0.59,0.20}
\definecolor{brown}{rgb}{0.69,0.31,0.31}
\definecolor{purple}{rgb}{0.38,0.18,0.81}
\definecolor{lightblue}{rgb}{0.1,0.57,0.7}
\definecolor{lightred}{rgb}{1,0.4,0.5}
\definecolor{cloudwhite}{cmyk}{0,0,0,0.025}


\usepackage{upquote}
\usepackage{listings}
% CSS
\lstdefinelanguage{CSS}{
keywords={color,background-image:,margin,padding,font,weight,display,position,top,left,right,bottom,list,style,border,size,white,space,min,width, transition:, transform:, transition-property, transition-duration, transition-timing-function}, 
sensitive=true,
morecomment=[l]{//},
morecomment=[s]{/*}{*/},
morestring=[b]',
morestring=[b]",
alsoletter={:},
alsodigit={-}
}

% Javascript
\lstdefinelanguage{Javascript}{
keywords={typeof, new, true, false, catch, function, return, null, catch, switch, var, if, in, while, do, else, case, break, this},
morecomment=[s]{/*}{*/},
morecomment=[l]//,
morestring=[b]",
morestring=[b]'
}

\lstdefinelanguage{HTML5}{
language=html,
sensitive=true, 
alsoletter={<>=-},  
morecomment=[s]{<!-}{-->},
tag=[s],
otherkeywords={
  % General
  >,
  % Standard tags
  <!DOCTYPE,
  </html, <html, <head, <title, </title, <style, </style, <link, </head, <meta, />,
  % body
  </body, <body,
  % Divs
  </div, <div, </div>, 
  % Paragraphs
  </p, <p, </p>,
  % scripts
  </script, <script,
  % More tags...
  <canvas, /canvas>, <svg, <rect, <animateTransform, </rect>, </svg>, <video, <source, <iframe, </iframe>, </video>, <image, </image>, <header, </header, <article, </article
  },
  ndkeywords={
  % General
  =,
  % HTML attributes
  charset=, src=, id=, width=, height=, style=, type=, rel=, href=, frameborder=
  % SVG attributes
  fill=, attributeName=, begin=, dur=, from=, to=, poster=, controls=, x=, y=, repeatCount=, xlink:href=,
  % properties
  margin:, padding:, background-image:, border:, top:, left:, position:, width:, height:, margin-top:, margin-bottom:, font-size:, line-height:,
  % CSS3 properties
  transform:, -moz-transform:, -webkit-transform:,
  animation:, -webkit-animation:,
  transition:,  transition-duration:, transition-property:, transition-timing-function:,
  }
  }

\lstdefinestyle{htmlcssjs} {%
  % General design
%  backgroundcolor=\color{editorGray},
basicstyle={\footnotesize\ttfamily},   
frame=b,
  % line-numbers
  xleftmargin={0.75cm},
  numbers=left,
  stepnumber=1,
  firstnumber=1,
  numberfirstline=true, 
  % Code design
  identifierstyle=\color{black},
  keywordstyle=\color{blue}\bfseries,
  ndkeywordstyle=\color{editorGreen}\bfseries,
  stringstyle=\color{editorOcher}\ttfamily,
  commentstyle=\color{brown}\ttfamily,
  % Code
  language=HTML5,
  alsolanguage=Javascript,
  alsodigit={.:;},  
  tabsize=2,
  showtabs=false,
  showspaces=false,
  showstringspaces=false,
  extendedchars=true,
  breaklines=true,
  % German umlauts
  literate=%
  {Ö}{{\"O}}1
  {Ä}{{\"A}}1
  {Ü}{{\"U}}1
  {ß}{{\ss}}1
  {ü}{{\"u}}1
  {ä}{{\"a}}1
  {ö}{{\"o}}1
  }
%
\lstdefinestyle{py} {%
language=python,
literate=%
*{0}{{{\color{lightred}0}}}1
{1}{{{\color{lightred}1}}}1
{2}{{{\color{lightred}2}}}1
{3}{{{\color{lightred}3}}}1
{4}{{{\color{lightred}4}}}1
{5}{{{\color{lightred}5}}}1
{6}{{{\color{lightred}6}}}1
{7}{{{\color{lightred}7}}}1
{8}{{{\color{lightred}8}}}1
{9}{{{\color{lightred}9}}}1,
basicstyle=\footnotesize\ttfamily, % Standardschrift
numbers=left,               % Ort der Zeilennummern
%numberstyle=\tiny,          % Stil der Zeilennummern
%stepnumber=2,               % Abstand zwischen den Zeilennummern
numbersep=5pt,              % Abstand der Nummern zum Text
tabsize=4,                  % Groesse von Tabs
extendedchars=true,         %
breaklines=true,            % Zeilen werden Umgebrochen
keywordstyle=\color{blue}\bfseries,
frame=b,
commentstyle=\color{brown}\itshape,
stringstyle=\color{editorOcher}\ttfamily, % Farbe der String
showspaces=false,           % Leerzeichen anzeigen ?
showtabs=false,             % Tabs anzeigen ?
xleftmargin=17pt,
framexleftmargin=17pt,
framexrightmargin=5pt,
framexbottommargin=4pt,
%backgroundcolor=\color{lightgray},
showstringspaces=false,      % Leerzeichen in Strings anzeigen ?
}%



%%========================================
%% Start of document
%%========================================
\begin{document}

\title{Spotify-ed: Music Recommendation and Discovery in Spotify}
\author{José Lage Bateira}

\thesisdate{31 de Janeiro de 2014}

\supervisor{Orientador}{Fabien Gouyon}

%% Uncomment committee stuff in the final version
% \committeetext{Aprovado em provas públicas pelo Júri:}
%\committeemember{Presidente}{Nome do presidente do júri}
%\committeemember{Arguente}{Nome do arguente do júri}
%\committeemember{Vogal}{Nome do vogal do júri}
%\signature

%% Specify cover logo (in folder ``figures'')
\logo{uporto-feup.pdf}

\additionalfronttext{Preparação da Dissertação}

%% Preliminary materials
\begin{Prolog}
  %!TEX root = ../mieic.tex

\chapter*{Resumo}

\chapter*{Abstract}

 % the abstract
  %!TEX root = ../report.tex

\chapter*{Acknowledgements}
%\addcontentsline{toc}{chapter}{Agradecimentos}

I would like to thank my beloved dog. He showed compassion, understanding and above all, unconditional love towards a ranty and tired friend.

\vspace{10mm}
\flushleft{José Lage Bateira}
  % the acknowledgments
  %!TEX root = ../mieic.tex

\cleardoublepage
% \thispagestyle{plain}

\vspace*{8cm}

\begin{flushright}
   \textsl{``Fancy quote here, \\
           that will blow your mind''} \\
\vspace*{1.5cm}
           By Someone really smart
\end{flushright}
    % initial quotation if desired
  \cleardoublepage
  \pdfbookmark[0]{Conteúdo}{contents}
  \tableofcontents
  \cleardoublepage
  \pdfbookmark[0]{Lista de Figuras}{figures}
  \listoffigures
  \cleardoublepage
  \pdfbookmark[0]{Lista de Tabelas}{tables}
  \listoftables
  %!TEX root = ../mieic.tex

\chapter*{Abreviaturas e Símbolos}
%\addcontentsline{toc}{chapter}{Abbreviations}
\chaptermark{ABREVIATURAS E SÍMBOLOS}

\begin{flushleft}
\begin{tabular}{l p{0.8\linewidth}}
API & Aplication Programming Interface

\end{tabular}
\end{flushleft}

  % the list of abbreviations used
\end{Prolog}

%% Body
\StartBody

%!TEX root = ../mieic.tex

\chapter{Introdução} \label{chap:intro}

Bem longe vão os tempos, antes da Internet, em que ouvir e descobrir música nova era um desafio por si só. Era necessário ir
Agora, com alguns cliques, temos acesso a um catálogo de música tão grande, que o nosso cérebro não consegue processar.

Existem dezenas de serviços online que oferecem isso mesmo.
Alguns especializam-se na criação/geração de playlists (que funcionam como rádios), outros em expandir o catálogo de música e outros focam-se mais na capacidade de sugestão e recomendação de música (ou artistas de música) aos utilizadores.
Estes últimos, apresentam as sugestões de artistas/álbuns/músicas ao utilizador de uma forma rudimentar como listas ou em grelha.

No entanto, listas ou grelhas não fornecem ao utilizador qualquer tipo de informação adicional sobre a relação entre os artistas nem justificam a sua semelhança \cite{Lamere2008}.
Até fazem parecer que não existe nenhuma relação/ligação entre os artistas recomendados, o que não é verdade.

Essas relações existem e podem ser representadas como uma rede de artistas interligados num grafo, onde cada nó é um artista de música, e cada ligação entre nós representa uma ligação forte de parecença entre os artistas.


\section{Contexto/Enquadramento} \label{sec:context}

\emph{Esta secção descreve a área em que o trabalho se insere, podendo
referir um eventual projeto de que faz parte e apresentar uma breve
descrição da empresa onde o trabalho decorreu.}

\section{Motivação e Objetivos} \label{sec:goals}

Apresenta a motivação e enumera os objetivos do trabalho terminando
com um resumo das metodologias para a prossecução dos objetivos.


\section{Projeto} \label{sec:proj}

Na continuação da secção anterior, e apenas no caso de ser um Projeto
e não uma Dissertação, esta secção apresenta resumidamente o projeto.


\section{Estrutura da Dissertação} \label{sec:struct}

Para além da introdução, esta dissertação contém mais x capítulos.
No capítulo~\ref{chap:sota}, é descrito o estado da arte e são
apresentados trabalhos relacionados.
%\todoline{Complete the document structure.}
No capítulo~\ref{chap:chap3}, é explicado em detalhe em que consiste o projeto
No capítulo~\ref{chap:chap4} praesent sit amet sem. 
No capítulo~\ref{chap:concl}  posuere, ante non tristique
consectetuer, dui elit scelerisque augue, eu vehicula nibh nisi ac
est. 
 
%!TEX root = ../mieic.tex

\chapter{Revisão Bibliográfica} \label{chap:chap2}

\section*{}

\section{Introdução}

Esta dissertação foca-se mais na forma como se apresenta o conteúdo que se pretende recomendar ao utilizador, e não qual o conteúdo que é sugerido (não obstante da sua importância obviamente).
No entanto, é quase impossível, no estudo do estado da arte, não se refirir outros projetos que se focam também no conteúdo.
Nesse sentido, será feita uma pequena análise dos conteúdos sugeridos.
Regra geral, os projetos que de seguida serão analisados, utilizam bases de dados externas, como o last.fm, para obter metadata que, convenientemente, também oferecem um tipo de recomendação de música com base numa pesquisa inicial.

\section{Projetos Relacionados} % (fold)
\label{sec:projetos_relacionados}


\subsection{liveplasma.com} % (fold)
\label{sub:projeto_1}

O liveplasma.com é uma aplicação em flash que mosta relações de artistas de música em forma de grafo, para além de também permitir criar grafos com livros e filmes.
Este não permite editar o grafo e, ao clicar num nó o grafo, é novamente gerado apartir desse nó.

\begin{figure}[tb]
  \begin{center}
    \includegraphics[width=\textwidth]{liveplasma.pdf}
  \end{center}
  \caption{liveplasma: resultado da pesquisa "Amália Rodrigues". Canto superior esquerdo: álbums da artista; Canto inferior esquerdo: \emph{mini-player} do youtube}.
  \label{fig:sota_liveplasma}
\end{figure}

Na figura \ref{fig:sota_liveplasma} podemos ver o resultado de uma pesquisa.
É possível ver a grelha com os álbums que o artista lançou, que redirecionam o utilizador para a Amazon\footnote{http://amazon.com} para comprar os álbums e um \emph{mini-player} que começa a reproduzir uma música do artista diretamente do Youtube.

É possível controlar que músicas são reproduzidas de uma forma interessante: ao passar o rato por cima de um nó, aparece dois botões que permitem reproduzir música só do próprio artista (botão \emph{only}) ou só de artistas parecidos (botão \emph{similar}).
É possível ver esses botões na figura \ref{fig:sota_liveplasma2}

\begin{figure}[tb]
  \begin{center}
    \includegraphics[]{liveplasma2.pdf}
  \end{center}
  \caption{liveplasma: interface para reprodução de música. Botão \emph{similar} reproduz músicas de artistas parecidos; Botão \emph{only} só reproduz músicas do artista pesquisado.}
  \label{fig:sota_liveplasma2}
\end{figure}

\subsubsection{Prós} % (fold)
\label{ssub:liveplasma_pros}

Os aspectos interessantes desta ferramenta são:

\begin{itemize}
  \item Links para compra dos álbums
  \item Reproduzir músicas de artistas semelhantes
\end{itemize}

% subsubsection pros (end)

\subsubsection{Contras} % (fold)
\label{ssub:liveplasma_contras}

O grafo desenhado é bastante confuso quando existem muitos nós com muitas ligações.
Isto acontece quando existem muitos artistas semelhantes.
Para além disso, são atribuídas cores aos nós que devem identificar o grau de parecença entre os artistas.
No entanto não existe nenhum tipo de informação que explique qual o seu verdadeiro significado ao utilizador, assim como também não existe uma explicação das ligações entre os nós.

É também de notar que o tamanho dos nós é diretamente proporcional à popularidade dos artistas respectivos, mas mais uma vez, este tipo de informação não é dada ao utilizador.

Outra falha a apontar é o facto de não se conseguir distinguir o nó de pesquisa dos restantes resultados em \ref{fig:sota_liveplasma} por exemplo.

% subsubsection contras (end)

\subsubsection{Resumo} % (fold)
\label{ssub:liveplasma_resumo}

Em suma, o liveplasma é usável, mas peca por ter muitas cores e ligações que tornam a experiência do utilizador ainda mais difícil do que a tradicional apresentação em lista ou grelha.

% subsubsection resumo (end)

% subsection projeto_1 (end)

\subsection{audiomap.tuneglue.net} % (fold)
\label{sub:projeto_2}



\subsubsection{Prós} % (fold)
\label{ssub:audiomap_pros}

% subsubsection pros (end)

\subsubsection{Contras} % (fold)
\label{ssub:audiomap_contras}

% subsubsection contras (end)

\subsubsection{Resumo} % (fold)
\label{ssub:audiomap_resumo}

% subsubsection resumo (end)

% subsection projeto_2 (end)

\subsection{Discovr.info} % (fold)
\label{sub:projeto_3}


% subsection projeto_3 (end)

% section projetos_relacionados (end)


\section{Resumo ou Conclusões}


%!TEX root = ../mieic.tex

\chapter{Projeto}
\label{chap:chap3}

\section*{}

O principal objectivo desta dissertação, como foi referido no capítulo \ref{chap:intro}, é desenvolver um (ou mais) módulo(s) de software que contribuam para uma melhoria na descoberta e recomendação de música num ambiente integrado entre o RAMA e o Spotify, por forma a tirar partido da representação gráfica do grafo de artistas de música do RAMA e da qualidade do serviço de \emph{Streaming} de música do Spotify.

Para tal, a proposta inicial desta dissertação consiste em desenvolver, no mínimo, um módulo que implemente uma das seguintes funcionalidades:

\begin{enumerate}
  \item \label{item:obj1} Integrar o serviço de \emph{streaming} de música do Spotify \textbf{no RAMA}
  \item \label{item:obj2} Integrar informação de um utilizador Spotify \textbf{no RAMA}
  \item \label{item:obj3} Melhorar design e funcionalidades \textbf{do RAMA}
  \item \label{item:obj4} Integrar a visualização de grafos de artistas de música \textbf{numa Aplicação Spotify}
  \item \label{item:obj5} Integrar o módulo de criação de \emph{playlists} do RAMA \textbf{numa Aplicação Spotify}
  \item \label{item:obj6} Integrar alguns dos módulos acima referidos \textbf{numa aplicação móvel}
\end{enumerate}

As três primeiras funcionalidades (\ref{item:obj1}, \ref{item:obj2} e \ref{item:obj3}) focam-se em melhorar o serviço do RAMA, usando API's do Spotify, ou seja, integrar o Spotify dentro do RAMA.
Por outro lado, as funcionalidades \ref{item:obj4} e \ref{item:obj5} têm como objetivo integrar o RAMA dentro do Spotify, através de uma Aplicação Spotify, que funciona como plugin do programa principal do Spotify.
A última funcionalidade (\ref{item:obj6}) teria de implementar algumas das anteriores num Sistema Operativo Móvel (Android, iOS ou Windows Phone).

Este capítulo procura analisar todas as condicionantes que afetam a escolha  dos módulos a desenvolver, e em que ambientes estes se encaixam melhor (Aplicação Spotify, aplicação móvel ou RAMA).

Inicialmente será explorado o ambiente de desenvolvimento do Spotify, seguido das tecnologias que suportam os referidos módulos, da arquitetura subjacente a cada módulo, e seguida de experimentações feitas que ajudaram na tomada de decisão final.

No final, deve ficar claro quais serão os módulos de software a desenvolver, tendo em conta que o seu objetivo é contribuir para uma melhoria na descoberta e recomendação de música num ambiente relacionado com o RAMA.


\section{Spotify} % (fold)
\label{sec:spotify}


% section spotify (end)

\section{Tecnologias} % (fold)
\label{sec:tecnologias}

% section tecnologias (end)

\section{Arquitetura} % (fold)
\label{sec:arquitetura}


% section arquitetura (end)

\section{Experimentação Feita} % (fold)
\label{sec:experimentacao}

% section experimentacao (end)

\section{Resumo e Conclusões}



%!TEX root = ../mieic.tex

\chapter{Plano de Trabalho}\label{chap:chap4}

\section*{}

Este capítulo pode ser dedicado à apresentação de detalhes de nível
mais baixo relacionados com o enquadramento e implementação das
soluções preconizadas no capítulo anterior.
Note-se no entanto que detalhes desnecessários à compreensão do
trabalho devem ser remetidos para anexos.

Dependendo do volume, a avaliação do trabalho pode ser incluída neste
capítulo ou pode constituir um capítulo separado.

\section{Task 1} % (fold)
\label{sec:task_1}
\lipsum
% section task_1 (end)

\section{Task 2} % (fold)
\label{sec:task_2}
\lipsum
% section task_2 (end)



\section{Resumo ou Conclusões}

\lipsum[1]

%!TEX root = ../report.tex

\chapter{Conclusions}
\label{chap:concl}

\section*{}


\section{Summary} % (fold)
\label{sec:summary}

% these were the proposed tests (check, check, check...)
% these were the results

% Material can be found in <url>

% section summary (end)

\section{Discussion} % (fold)
\label{sec:discussion}

  % results discussion

% section discussion (end)

\section{Future Work} % (fold)
\label{sec:future_work}

% more tests (and more segmented)
% use more external databases to improve user experience
% all of the other features from the feedback form
% and the smc group

% section future_work (end)

%% Final materials

%% Bibliography

\renewcommand{\bibname}{Referências}

\bibliography{bib/myrefs}
\bibliographystyle{bib/unsrt-pt}

% \renewcommand{\bibname}{Referências}
% \PrintBib{./bib/mieic}

%% Comment next 2 commands if numbered appendices are not used
% \appendix
% \include{appendix1}

% \PrintIndex

\end{document}
