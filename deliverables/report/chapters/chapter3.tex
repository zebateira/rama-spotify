%!TEX root = ../mieic.tex

\chapter{Context and Methodologies}
\label{chap:chap3}

\section*{}

The primal objective of this dissertation, as referred in chapter \ref{chap:intro}, is to develop one or more software modules that will improve Spotify Users' music discovery and recommendation experience using visual tools to represent the music artists' relations and Spotify's high quality music streaming service.

The initial proposal was to develop a module that implements, at least, one of the following features:

\begin{enumerate}
  \item \label{item:obj1} Integrate Spotify's music stream into RAMA's website
  \item \label{item:obj2} Integrate information from the Spotify user into RAMA
  \item \label{item:obj3} Melhorar design e funcionalidades \textbf{do RAMA}
  \item \label{item:obj4} Integrar a visualização de grafos de artistas de música \textbf{numa Aplicação Spotify}
  \item \label{item:obj5} Integrar o módulo de criação de \emph{playlists} do RAMA \textbf{numa Aplicação Spotify}
  \item \label{item:obj6} Integrar alguns dos módulos acima referidos \textbf{numa aplicação móvel}
\end{enumerate}

As três primeiras funcionalidades (\ref{item:obj1}, \ref{item:obj2} e \ref{item:obj3}) focam-se em melhorar o serviço do RAMA, usando API's do Spotify, ou seja, integrar o Spotify dentro do RAMA.
Por outro lado, as funcionalidades \ref{item:obj4} e \ref{item:obj5} têm como objetivo integrar o RAMA dentro do Spotify, através de uma Aplicação Spotify, que funciona como \emph{plugin} do programa principal do Spotify.
A última funcionalidade (\ref{item:obj6}) teria de implementar algumas das anteriores num Sistema Operativo Móvel (Android, iOS ou Windows Phone).

Este capítulo procura analisar todas as condicionantes que afetam a escolha  dos módulos a desenvolver, e em que ambientes estes se encaixam melhor (Aplicação Spotify, aplicação móvel ou RAMA).

Inicialmente será explorado o ambiente de desenvolvimento que o Spotify disponibiliza, ou seja, que tecnologias tem disponíveis para \emph{developers}.
De seguida serão analisadas quais dessas tecnologias assentam melhor em cada um dos módulos propostos a desenvolver, através de experimentações feitas, e quando necessário, será descrito um possível esquema de arquitetura por forma a facilitar a explicação do problema.


No final deste capítulo, deve ficar claro quais serão os módulos de software a desenvolver, que tecnologias irão ser usadas e qual o esquema geral da sua arquitetura.
O produto final deve de ir ao encontro do objetivo de contribuir para uma melhoria na descoberta e recomendação de música num ambiente relacionado com o RAMA.


\section{Introducing Spotify} % (fold)
\label{sec:spotify}

  O Spotify é um serviço de \emph{streaming} de música que permite ouvir, através de uma ligação de Internet, qualquer música que o Spotify possua no seu catálogo.


  \subsection{Ferramentas de Desenvolvimento} % (fold)
  \label{sub:ferramentas_de_desenvolvimento}
  
    No momento de escrita deste relatório, o Spotify tem disponível um conjunto de ferramentas\footnote{http://developer.spotify.com/technologies} para desenvolver módulos de software que podem estar embebidos nas mais diversas aplicações (\emph{third-party applications}) ou então dentro do \emph{Spotify Desktop Client}.

    Existem quatro ferramentas de desenvolvimento, cada uma delas com o seu propósito e utilidade.


    \subsubsection{Spotify Apps} % (fold)
    \label{ssub:spotify_apps}
      Serve para desenvolver Aplicações Spotify\footnote{https://developer.spotify.com/technologies/apps} que são usadas pelos utilizadores Spotify dentro do \emph{Spotify Desktop Client}. Estas são aplicação \emph{HTML5}\footnote{http://www.w3.org/TR/html5/}.

      \begin{figure}
        \begin{center}
          \includegraphics[width=\textwidth]{spotify.pdf}
        \end{center}
        \caption{Spotify: interface do modo de descoberta do \emph{desktop client}}
        \label{fig:spotify_apps}
      \end{figure}

      Na figura \ref{fig:spotify_apps} é possível ver o aspeto do \emph{Spotify Desktop Client}.
      Na barra lateral esquerda, dentro do separador \emph{Apps}, aparece a lista de aplicações já instaladas, assim como o \emph{App Finder}, que permite procurar e instalar aplicações com apenas um clique.
 
      Na figura \ref{fig:spotify_apps2} está aberta a aplicação da Last.fm. É possível ver que as Aplicações Spotify têm apenas um espaço reservado embutido no \emph{Spotify Desktop Client}.

      \begin{figure}
        \begin{center}
          \includegraphics[width=\textwidth]{spotify_apps.pdf}
        \end{center}
        \caption{Spotify: Aplicação Last.fm aberta no \emph{Spotify Player}}
        \label{fig:spotify_apps2}
      \end{figure}

      Para o seu desenvolvimento destas aplicações são disponibilizadas duas \emph{frameworks}: \emph{API Framework}\footnote{https://developer.spotify.com/docs/apps/api/1.0/} e \emph{Views Framework}\footnote{https://developer.spotify.com/docs/apps/views/1.0/}.
      A primeira fornece uma interface para recolher metadados de artistas, álbuns e músicas e controlar o reprodutor de música.
      A segunda fornece componentes de design como botões, listas, abas, entre outros, para o desenvolvimento da aplicação.

      Para desenvolver os módulos \ref{item:obj4} e \ref{item:obj5} esta é a ferramenta mais apropriada.

    % subsubsection spotify_apps (end)


    \subsubsection{Spotify Widgets} % (fold)
    \label{ssub:spotify_widgets}
      As Widgets\footnote{https://developer.spotify.com/technologies/widgets} são pequenos componentes que se podem embeber em \emph{websites}.
      No momento da escrita deste relatório existem dois componentes: \emph{Play Button} (\ref{fig:spotify_play_button}) e \emph{Follow Button} (\ref{fig:spotify_follow_button}).

      \begin{figure}
        \begin{center}
          \includegraphics{spotify_play_button.pdf}
        \end{center}
        \caption{Spotify: \emph{Play Button} pode ser embebido em \emph{websites}.}
        \label{fig:spotify_play_button}
      \end{figure}

      \begin{figure}
        \begin{center}
          \includegraphics{spotify_follow_button.pdf}
        \end{center}
        \caption{Spotify: \emph{Follow Button} permite seguir um artista.}
        \label{fig:spotify_follow_button}
      \end{figure}

      No entanto, existe algumas limitações no uso destas componentes.
      No Spotify, apenas utilizadores que tenham criado conta no serviço Spotify é que podem usar o mesmo.
      O mesmo também se aplica a estas \emph{widgets} - apesar de estas existirem numa aplicação externa ao Spotify, apenas utilizadores Spotify podem usá-las.
      Esta limitação pode fazer sentido para o \emph{Follow Button}, mas o \emph{Play Button} torna-se inútil para utilizadores que não usem o Spotify.
      Outro problema surge quando a música do \emph{Play Button} não está disponível no País em que o utilizador está.

      Estas \emph{widgets} apenas servem de hiperligação ou ao \emph{Player} do Spotify ou ao \emph{WebPlayer} do Spotify.
      Na realidade, usando estas \emph{widgets}, o stream de música do Spotify é sempre reproduzido dentro do ambiente do Spotify, e nunca em aplicações externas. \\

      Para embeber as \emph{widgets} apenas é necessário introduzir um elemento \emph{iframe} no código \emph{HTML} fonte do \emph{website}:

      \lstinputlisting[language=HTML,caption={Código \emph{HTML} para embeber um \emph{Play Button}}]{snippets/play_button.html}

      As \emph{widgets} seriam úteis para desenvolver os módulos \ref{item:obj1} e \ref{item:obj3}.

    % subsubsection spotify_widgets (end)

    \subsubsection{Libspotify SDK} % (fold)
    \label{ssub:libspotify_sdk}
    
      Libspotify SDK\footnote{https://developer.spotify.com/technologies/libspotify} é uma API que permite adicionar os serviços do Spotify em aplicações externas.
      No entanto, existem algumas limitações para os utilizadores destas aplicações.
      
      Existem, dois tipos de conta a que o utilizador pode subscrever: conta grátis e conta \emph{premium}.
      Como foi referido anteriormente (\ref{ssub:spotify_widgets}) apenas utilizadores Spotify podem interagir com qualquer componente do Spotify, dentro ou fora das aplicações nativas do mesmo.
      Libspotify fornece uma interface que permite a um utilizador fazer \emph{login} no Spotify em aplicações externas por forma a poder ouvir música do Spotify, criar playlists e outras funcionalidades.
      No entanto, os únicos utilizadores que pode fazer \emph{login} nestas aplicações que usam Libspotify, são utilizadores \emph{premium}.
      Para além de que, os \emph{developers} da própria aplicação também precisam de ser utilizadores \emph{premium}.

      Neste sentido, uma aplicação que, para funcionar, necessita de que o utilizador, para além de possuir uma conta Spotify, também pague uma subscrição mensal \emph{premium}, é uma aplicação bastante restritiva.

      Esta ferramenta pode ser usada para desenvolver os módulos \ref{item:obj1}, \ref{item:obj2} e \ref{item:obj6}.
      

    % subsubsection libspotify_sdk (end)


    \subsubsection{Metadata API} % (fold)
    \label{ssub:metadata_api}
    
      A \emph{Metadata API}\footnote{https://developer.spotify.com/technologies/web-api} disponibiliza publicamente informação de músicas, álbuns e artistas da Base de dados do Spotify.

      Através de pedidos HTTP é possível obter informação da base de dados do Spotify. Existe dois tipos de pedidos que esta API disponibiliza: \emph{search}\footnote{https://developer.spotify.com/technologies/web-api/search} e \emph{lookup}\footnote{https://developer.spotify.com/technologies/web-api/lookup}.
      Para obter informação detalhada de, por exemplo, um artista, é necessário saber o deu identificador único.
      Esse identificador é um \emph{URI} da forma:

      \url{spotify:artist:<artist_id>}, onde \emph{artist\_id} é um identificar único.

      Exemplo:

      \url{spotify:artist:65nZq8l5VZRG4X445F5kmN}, é o identificador único da fadista "Mariza". \\

      Também existem identificadores únicos para álbuns:

      \url{spotify:album:5d1LpIPmTTrvPltx26TlEU} (álbum "Fado Tradicional" de "Mariza") \\

       e para faixas de música:

       \url{spotify:track:2vqYasauhDLVjTt7CGWK6y} (música "Fado Vianinha" do mesmo álbum) \\

      Para obter este \emph{URI} é preciso interrogar a base de dados com um método de pesquisa.
      Para isso, usa-se o \emph{search}.

      \begin{description}
        \item[\emph{Search}] \hfill

          O \emph{URL} base de utilização é:

          \url{http://ws.spotify.com/search/1/album}, para pesquisa de álbuns.

          Se se pretender pesquisar Artistas, usa-se \emph{artist}, se se pretender pesquisar Faixas de música, usa-se \emph{track}. \\

          Exemplos:

          \url{http://ws.spotify.com/search/1/album?q=foo} \\
          \url{http://ws.spotify.com/search/1/artist.json?q=red+hot} \\

          O resultado da \emph{query}, por defeito, tem o formato \emph{XML}. No entanto, também se pode especificar o formato \emph{JSON} (como no segundo exemplo).

          Dada a \emph{query}: \\
          \url{http://ws.spotify.com/search/1/artist.json?q=camane} (fadista "Camané")

          Obtém-se o resultado:

          \lstinputlisting[caption={Os resultados são ordenados pelo atributo "popularity"}]{snippets/search_camane.json}

        \item[\emph{Lookup}] \hfill \\
          Depois de obtido o \emph{URI} identificador, é possível obter mais informações de um conteúdo usando o \emph{lookup}.

          A seguinte \emph{query}: \\
          \url{http://ws.spotify.com/lookup/1/.json?uri=spotify:artist:3MLPFTe4BrpEV2eOVG0gLK}

          Retorna:

          \lstinputlisting[caption={Resultado do \emph{lookup} do fadista "Camané"}]{snippets/lookup_camane.json}

      \end{description}

      Esta API seria bastante útil para desenvolver qualquer um dos seis módulos propostos.
      Aliás, até complementa as \emph{Widgets} e o \emph{Libspotify SDK}.

    % subsubsection metadata_api (end)

  % subsection ferramentas_de_desenvolvimento (end)

  \subsection{Experimentações Feitas} % (fold)
  \label{sub:experimentacoes}
  
    Numa primeira experiência com as ferramentas, foi criado um pequeno \emph{website} que permite pesquisar e ouvir Música do Spotify usando a \emph{Metadata API} e \emph{Spotify Widgets}: \\

    \url{http://carsy.github.io/spotify-playground} \\

    Na figura \ref{fig:playground} é possível ver o resultado de uma pesquisa, e a \emph{Widget Play Button} com o resultado selecionado da pesquisa.

    \begin{figure}
      \centering

      \begin{subfigure}[b]{0.38\textwidth}
        \includegraphics[width=\textwidth]{playground.pdf}
        \caption{Resultado da pesquisa "Mariza"}
        \label{fig:playgroun_a}
      \end{subfigure}

      \begin{subfigure}[b]{0.38\textwidth}
        \includegraphics[width=\textwidth]{playground2.pdf}
        \caption{Depois de selecionado o álbum "Fado Tradicional" aparece o \emph{Play button} com as faixas do álbum.}
        \label{fig:playground_b}
      \end{subfigure}

      \caption{Experiência com \emph{Metadata API} e \emph{Play Button Widget} (código fonte: \url{github.com/carsy/spotify-playground})}
      \label{fig:playground}

    \end{figure}

    Verificou-se que as duas ferramentas estão bem documentadas e em constante atualização. \\

    Outra experiência foi realizada para verificar se é possível usar o elemento \emph{canvas} numa Aplicação Spotify.
    Isto é necessário pois será a única forma de poder desenhar graficamente o grafo.
    Para isso foi apenas necessário criar uma aplicação com o seguinte código fonte:

    \begin{lstlisting}[caption={Elemento \emph{iframe} que embebe o \emph{website} do RAMA na aplicação}]
      <iframe src="http://rama.inescporto.pt/app" frameborder="0"></iframe>\end{lstlisting}

    Desta forma, é possível embeber o RAMA na Aplicação Spotify (que usa o elemento \emph{canvas} para desenhar o grafo).
    Resultado final na figura \ref{fig:rama_spotifyed}.

    \begin{figure}
      \begin{center}
        \includegraphics[width=\textwidth]{rama.pdf}
      \end{center}
      \caption{\emph{Website} do RAMA embebido numa Aplicação Spotify}
      \label{fig:rama_spotifyed}
    \end{figure}

    Apesar de \emph{iframes} serem suportadas, existem outros componentes que não o são.
    A aplicação não é usável, pois não permite, por exemplo, reproduzir automaticamente faixas de artistas.

    No entanto existe uma forma de testar quais os elementos de \emph{HTML5} suportados, usando uma aplicação interna do Spotify.
    Na figura \ref{fig:canvas_support} é possível ver que o elemento \emph{canvas} é suportado a cem por cento.

    \begin{figure}
       \begin{center}
         \includegraphics[width=0.5\textwidth]{canvas_support.pdf}
       \end{center}
       \caption{Resultado do teste do elemento \emph{canvas}}
       \label{fig:canvas_support}
     \end{figure}

  % subsection experimentacoes (end)

  \subsection{Conclusão} % (fold)
  \label{sub:conclusao}
  
    A prova de conceito desenvolvida (\ref{fig:rama_spotifyed}) demonstrou-se a mais indicada para o objetivo final de criar um ambiente integrado entre o Spotify e o RAMA.

    Assim, os módulos a serem desenvolvidos são \ref{item:obj4} e \ref{item:obj5}.

  % subsection conclusao (end)

% section spotify (end)

\section{Technologies used} % (fold)
\label{sec:technologies}

  As seguintes tecnologias serão utilizadas nas fase de desenvolvimento, testes e otimização da Aplicação Spotify.

  \subsection{\emph{Spotify Desktop Client}} % (fold)
  \label{sub:subsection_name}
    O desenvolvimento de Aplicações Spotify é feito de forma integrada no programa.

    Para abrir uma Aplicação Spotify, localmente, escreve-se o seguinte na barra de pesquisa: spotify:app:rama

    Onde \emph{rama} deve ser o identificador da aplicação declarado no ficheiro \emph{manifest.json}\footnote{ficheiro situado na \emph{root} da pasta do projeto}. \\
    Exemplo de ficheiro \emph{manifest.json}:

    \lstinputlisting[caption={manifest.json: \emph{BundleIdentifier} é o identificador da aplicação; \emph{Dependencies} declara as dependências das API's necessárias ao desenvolvimento.}]{snippets/manifest.json}

    Existem outras opções úteis a que se pode aceder usando a tab \emph{Develop} (\ref{fig:html5_support}).
    A opção "Show Inspector" abre a janela \emph{Webkit Development Tools} (\ref{sub:webkit_tools})

    \begin{figure}
      \begin{center}
        \includegraphics[width=0.6\textwidth]{html5_support.pdf}
      \end{center}
      \caption{Menu \emph{Develop}}
      \label{fig:html5_support}
    \end{figure}
  
  % subsection subsection_name (end)

  \subsection{Webkit Development Tools - webkit.org} % (fold)
  \label{sub:webkit_tools}

    A partir do \emph{webkit}, tem-se acesso a várias ferramentas úteis para o desenvolvimento \emph{web} (\ref{fig:webkit_inspector}).

    \begin{figure}
      \begin{center}
        \includegraphics[width=\textwidth]{webkit_inspector.pdf}
      \end{center}
      \caption{Webkit: Vista da tab \emph{Inspector}. Outas ferramentas disponíveis (tabs): \emph{Resources, Network, Sources, Timeline, Profiles, Audits} e \emph{Console}.}

      \label{fig:webkit_inspector}
    \end{figure}

    A mais importantes são:

    \begin{description}
      \item[Inspector] Permite inspecionar e editar o código \emph{HTML} e \emph{CSS} da aplicação diretamente  (\ref{fig:webkit_inspector}).
      \item[Network] Permite, por exemplo, ver o tempo que cada componente da aplicação demorou a carregar (uma imagem ou um ficheiro \emph{css}) (\ref{fig:webkit_network}).
      \item[Profile] Permite identificar que partes do código \emph{javascript} são as mais frequentemente executadas (\ref{fig:webkit_profile}).
      \item[Audit] Ajuda a perceber quantos recursos estão a ser descarregados desnecessariamente, como por exemplo, regras de \emph{CSS} que não estão a ser usadas (\ref{fig:webkit_audit}).
      \item[Console] Muito útil para \emph{debug} de \emph{javascript}.

    \end{description}


    \begin{figure}
      \begin{center}
        \includegraphics[width=\textwidth]{webkit_network.pdf}
      \end{center}
      \caption{Webkit Network}
      \label{fig:webkit_network}
    \end{figure}

    \begin{figure}
      \begin{center}
        \includegraphics[width=\textwidth]{webkit_profile.pdf}
      \end{center}
      \caption{Webkit Profile: É possível ver que a renderização do grafo é o que ocupa mais tempo de processamento como esperado. No entanto, existe uma parte de \emph{JQuery} que ocupa 12.72\% do tempo de processamento, o que pode indicar um possível ponto de melhoria de performance.}
      \label{fig:webkit_profile}
    \end{figure}

    \begin{figure}
      \begin{center}
        \includegraphics[width=\textwidth]{webkit_audit.pdf}
      \end{center}
      \caption{Webkit Audit: 96\% do código \emph{CSS} não está a ser usado, sendo por isso, um ponto de melhoria reduzir a quantidade de informação descarregada.}
      \label{fig:webkit_audit}
    \end{figure}

    \begin{figure}
      \begin{center}
        \includegraphics[width=\textwidth]{webkit_console.pdf}
      \end{center}
      \caption{Webkit Console: Erros de \emph{Javascript} aparecem destacados para chamar a atenção.}
      \label{fig:webkit_console}
    \end{figure}

  % subsection webkit_tools (end)

  \subsection{Npmjs - npmjs.org} % (fold)
  \label{sub:npm}
    Gestor de pacotes de software e dependências.
    Para usar \emph{npm} é necessário um ficheiro de configuração \emph{package.json} que permite identificar quais os pacotes de que a aplicação depende, e as suas versões. \\
    Exemplo:

    \lstinputlisting[caption={\emph{package.json}: ao indicar a versão com "*", significa que se deve usar sempre a mais recente.}]{snippets/package.json}

  % subsection npm (end)

  \subsection{Gruntjs - gruntjs.com} % (fold)
    \label{sub:gruntjs} 
      Programa de gestão de tarefas automatizadas.
      Muito útil para testes, compilação e otimização de código.
      É possível por exemplo, quando qualquer parte do código mudar, a aplicação automaticamente atualiza com as mudanças mais recentes, sem ser preciso refrescar manualmente a aplicação.
  % subsection gruntjs (end)

  \subsection{Arborjs - arborjs.org} % (fold)
  \label{sub:arborjs}
    Framework de javascript para desenho de grafos. Foi já utilizada no desenvolvimento do RAMA (existe sempre a possibilidade de se usar outra ferramenta substituta caso esta não for adequada).
  
  % subsection arborjs (end)

% section technologies (end)

\section{Conclusions}

  A escolha final do módulo a desenvolver é a Aplicação Spotify.

  Apesar de as outras opções serem também viáveis, a possibilidade de poder integrar uma interface estilo RAMA num ambiente que os utilizadores já se sentem confortáveis (Spotify), é muito favorável a que seja melhor aceite pelos mesmos.

  É esperado que as tecnologias a usar ajudem no desenvolvimento desta dissertação.

  Em suma, será desenvolvida uma Aplicação Spotify que implemente os módulos \ref{item:obj4} e \ref{item:obj5}.